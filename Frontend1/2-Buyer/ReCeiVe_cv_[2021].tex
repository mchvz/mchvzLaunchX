%%%%%%%%%%%%%%%%%%%%%%%%%%%%%%%%%%%%%%%%%
% ReCeiVe
% LuaLaTeX Template
% Version 1.12.0 (10/10/2020)
%
% Original authors:
% Ged Lex (gedlex@hotmail.ch)
%
% Important note:
% This template must be compiled with LuaLaTeX, the below lines will ensure this
%!TEX TS-program = lualatex
%!TEX encoding = UTF-8 Unicode
%%%%%%%%%%%%%%%%%%%%%%%%%%%%%%%%%%%%%%%%%

%----------------------------------------------------------------------------------------
%	PACKAGES AND OTHER DOCUMENT CONFIGURATIONS
%----------------------------------------------------------------------------------------
\documentclass[rightPos]{ReCeiVe}      % By default, use 'letterpaper' for US letter
%\geometry{letterpaper, % paper size
%          left=1.75cm, right=1.75cm, top=1.5cm, bottom=1.5cm, footskip=.6cm, headsep=.5cm, % margins
%          showframe}                  % show geometry frame
%          heightrounded}              % avoid underfull vbox warning
          
% Color for highlights
%\definecolor{highlight}{HTML}{EB9534} % Specify your own color
%\colorlet{highlight}{white}           % Set predefined color
% Default colors include: darkgray, gray, lightgray, lightblue, orange, red, concrete

\renewcommand{\baselinestretch}{1.5}
\definecolor{Website}{HTML}{0099FF}
\definecolor{Github}{HTML}{171515}
\definecolor{Linkedin}{HTML}{0e76a8}
\definecolor{Skype}{HTML}{00aff0}
\definecolor{Twitter}{HTML}{00acee}
\definecolor{Youtube}{HTML}{c4302b}
\definecolor{Facebook}{HTML}{3b5998}
\definecolor{Link}{HTML}{3366BB}


% Colors for text - uncomment and modify
%\definecolor{darktext}{HTML}{414141}
%\definecolor{text}{HTML}{414141}
%\definecolor{graytext}{HTML}{414141}
%\definecolor{lighttext}{HTML}{414141}

%----------------------------------------------------------------------------------------
%	PERSONAL INFORMATION
%	Comment any of the lines below if they are not required
%----------------------------------------------------------------------------------------
\background{pics/background2.pdf}
\photo[circle,edge,fill,left]{2.5cm}{pics/pp2.pdf}
\name{Diana}{Avendaño}
\address{Jalisco, MX}
\mobile{+52 12 3456 7890}
%\email{dianaavendanovazquez@}
%\github{gedlex}
%\linkedin{linkedin}
%\homepage{www.homepage.com}
%\twitter{@twit}
%\xing{xing name}
%\stackoverflow{SOid}{SOname}
%\skype{skypeid}
%\reddit{reddit account}
%\extrainfo{info}

\position{BSc in Mechanical Engineering} % Your expertise/fields
\headwords{Engineer, software geek, technical enthusiast and carpenter} % A few headwords to describe yourself
\quote{"Some men see things as they are, and ask why. I dream of things that never were, and ask why not." - Robert Kennedy} % A quote or statement

%----------------------------------------------------------------------------------------
%	SIDEBAR CONTENT
%	Fill in the information you would like to add to your sidebar
%----------------------------------------------------------------------------------------

\aboutMe{Engineer, software geek, technical enthusiast and carpenter}
\skillset[label/ \hfill ]{MATLAB/.8,Python/.6,Java/.8, SQL/.8, JavaScript/.4}
\languages{{English/Fluent (C1 Certificate)},{Spanish/Mother tongue},{German/Conversational}}


% Usage: \userSection[<section title>]{<content>}
% \userSection[title]{content}

%----------------------------------------------------------------------------------------
\begin{document}
% Print the header
% Usage: \makecvheader[<position>]
\makecvheader[L]
% Print the sidebar
% Usage: \makecvsidebar[xOffset/value,yOffset/value,noRadius|radius]
\makecvsidebar

%----------------------------------------------------------------------------------------
%	CV/RESUME CONTENT
%	Each section is imported separately, open each file in order to modify it
%----------------------------------------------------------------------------------------
% Remove space before first section
\vspace{-\acvSectionTopSkip}

\section{Education}
\cventry
{University of Guadalajara} % Education Title
{Master of Physics} % Organization
{Jalisco, MX} % Location
{08/2019 - 07/2021} % Date

\section{Position}
\cventry
{MyCompany Name} % Education Title
{Software Engineer - BackEnd} % Organization
{Jalisco, MX} % Location
{08/2021 - } % Date

\section{Manager}
\cventry
{MyCompany Name} % Education Title
{John Doe} % Organization
{Jalisco, MX} % Location
{08/2020 - } % Date


\section{Contact me!!}

.\\[.3cm]
\begin{tabular}{r l}
	
	\multicolumn{2}{c}{
		\href{https://www.yourwebsite.com/}{\textcolor{Website}{\huge{\faWordpress}}}
		\quad
		\href{https://github.com/}{\textcolor{Github}{\huge{\faGithub}}}
		\quad
		\href{https://www.linkedin.com/}{\textcolor{Linkedin}{\huge{\faLinkedinSquare}}}
		\quad
		\href{https://www.skype.com/}{\textcolor{Skype}{\huge{\faSkype}}}
		\quad
		\href{https://twitter.com/}{\textcolor{Twitter}{\huge{\faTwitter}}}
		\quad
		\href{https://www.youtube.com/}{\textcolor{Youtube}{\huge{\faYoutubePlay}}}
		\quad
		\href{https://www.facebook.com/}{\textcolor{Facebook}{\huge{\faFacebookSquare}}}
	}
\end{tabular}
%\cventry
%{Licensed Carpenter} % Job Title
%{Timber Work GmbH} % Organization
%{Turicum, CH} % Location
%{08/2012 - 08/2017} % Date
%\begin{cvitems}
%\item {General timber construction works}
%\item {Tutoring of an apprentice}
%\end{cvitems}
%\section{Honors}
\begin{cvhonors}

\cvhonor
{1st Place} % Position
{Circular Economy Challenge} % Event
{Turicum, CH} % Location
{2021} % Date

\end{cvhonors}

%----------------------------------------------------------------------------------------
% Usage: \makecvfooter(<left>}{<center>}{<right>)
\makecvfooter{\today}{Ged Lex~~~·~~~CV}{\LaTeX{}}
\end{document}

